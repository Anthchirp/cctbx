\documentclass[11pt]{article}
\usepackage{cctbx_preamble}
\usepackage{amscd}

\title{Restraint Gradients}
\author{\rjgildea}
\date{\today}

\begin{document}
\maketitle

\section{ADP similarity restraint}
\label{ADP:similarity}
The anisotropic displacement parameters of two atoms are restrained to have the
same $U_{ij}$ components.  This is equivalent to a SHELXL SIMU restraint \cite{SHELX:man97}.
The weighted least-squares residual is defined as
\begin{equation}
R = w \sum_{i=1}^3 \sum_{j=1}^3 (U_{A,ij} - U_{B,ij})^2,
\end{equation}
which we note is the square of the Frobenius norm of the matrix of deltas.
But since $\mat{U}$ is symmetric, i.e. $U_{ij} = U_{ji}$, this can be rewritten as
\begin{equation}
R = w \left( \sum_{i=1}^3 (U_{A,ii} - U_{B,ii})^2 + 2 \sum_{i < j} (U_{A,ij} - U_{B,ij})^2 \right) .
\end{equation}
Therefore the gradient of the residual with respect to the diagonal element $U_{A,ii}$ is then
\begin{equation}
\partialder{R}{U_{A,ii}} = 2w(U_{A,ii} - U_{B,ii}).
\end{equation}
Similarly the gradient with respect to the off-diagonal element $U_{A,ij}$ is
\begin{equation}
\partialder{R}{U_{A,ij}} = 4w(U_{A,ij} - U_{B,ij}).
\end{equation}

\section{Rigid-bond restraint}

In a `rigid-bond' restraint the components of the anisotropic displacement parameters
of two atoms in the direction of the vector connecting those two atoms are restrained
to be equal.  This corresponds to Hirshfeld's `rigid-bond' test \cite{Hirshfeld:1976} for testing
whether anisotropic displacement parameters are physically reasonable (see SHELX
manual, DELU restraint \cite{SHELX:man97}).  We must therefore minimise the mean square displacement of
the atom in the direction of the bond.

The weighted least-squares residual is then
\begin{equation}
R = w(z^2_{A,B} - z^2_{B,A})^2,
\end{equation}
where in the Cartesian coordinate system the mean square displacement of atom A
along the vector $\overrightarrow{AB}$, $z^2_{A,B}$, is given by
\begin{equation}
z^2_{A,B} = \frac{\vec{r}^t\mat{U}_{cart,A}\vec{r}}{\norm{\vec{r}}^2},
\end{equation}
where
\begin{equation}
\vec{r} = \begin{pmatrix} x_A - x_B\\y_A - y_B\\z_A - z_B \end{pmatrix}
= \begin{pmatrix} x\\y\\z \end{pmatrix},
\end{equation}
$\vec{r}^t$ is the transpose of $\vec{r}$ (\textit{i.e.} a row vector) and
$\norm{\vec{r}}$ is the length of the vector $\overrightarrow{AB}$.

The derivative of the residual with respect to an element of $\vec{U}_{cart,A}$,
$U_{A,ij}$ is given by (using the chain rule)
\begin{align}
\partialder{R}{U_{A,ij}} &= \partialder{R}{z^2_{A,B}} \partialder{z^2_{A,B}}{U_{A,ij}}\\
&=2w(z^2_{A,B} - z^2_{B,A}) \partialder{z^2_{A,B}}{U_{A,ij}}\label{eqn:r_derivative}
\end{align}

The matrix multiplication in obtaining $z^2_{A,B}$ can be evaluated as follows
(remembering $\vec{U}_{cart}$ is symmetric):
\begin{align}
\vec{r}^t\vec{U}_{cart,A}\vec{r} &= 
\begin{pmatrix} x & y & z \end{pmatrix}
\begin{pmatrix} U_{11} & U_{12} & U_{13}\\
  U_{12} & U_{22} & U_{23}\\
  U_{13} & U_{23} & U_{33}\end{pmatrix}
\begin{pmatrix} x\\y\\z\end{pmatrix}\\
&= U_{11}\: x^2 + U_{22}\: y^2 + U_{33}\: z^2 + 2U_{12}\: xy + 2U_{13}\: xz + 2U_{23}\: yz
\end{align}
It then follows that
\begin{equation}
\partialder{z^2_{A,B}}{U_{11}} = \frac{x^2}{\norm{\vec{r}}^2} ,\qquad
\partialder{z^2_{A,B}}{U_{22}} = \frac{y^2}{\norm{\vec{r}}^2} ,\qquad
\partialder{z^2_{A,B}}{U_{33}} = \frac{z^2}{\norm{\vec{r}}^2} ,
\end{equation}
and
\begin{equation}
\partialder{z^2_{A,B}}{U_{12}} = \frac{2xy}{\norm{\vec{r}}^2} ,\qquad
\partialder{z^2_{A,B}}{U_{13}} = \frac{2xz}{\norm{\vec{r}}^2} ,\qquad
\partialder{z^2_{A,B}}{U_{23}} = \frac{2yz}{\norm{\vec{r}}^2} .
\end{equation}
These can be combined with \eqnref{r_derivative} to give us the derivatives
with respect to each $U_{ij}$ component.

\section{Isotropic ADP restraint}
Here we minimise the difference between the Cartesian ADPs, $\mat{U}_{cart}$ and
the isotropic equivalent, $\mat{U}_{eq}$.  As in section \ref{ADP:similarity}, we
must remember that we are dealing with symmetric matrices, and we can therefore
define the weighted least-squares residual as
\begin{equation}
R = w \left( \sum_{i=1}^3 (U_{ii} - U_{eq,ii})^2 + 2 \sum_{i<j} (U_{ij} - U_{eq,ij})^2 \right) ,
\end{equation}
where
\begin{equation}
\mat{U}_{eq} = 
\begin{pmatrix} U_{iso} & 0 & 0\\
  0 & U_{iso} & 0\\
  0 & 0 & U_{iso}\end{pmatrix},
\end{equation}
and
\begin{equation}
U_{iso} = \tfrac{1}{3} \mathrm{tr}(\mat{U}_{cart}).
\end{equation}
We expand the summation of the residual as follows
\begin{equation}
R = w \left( (U_{11} - U_{iso})^2 + (U_{22} - U_{iso})^2 + (U_{33} - U_{iso})^2 + 2 U_{12}^2 + 2 U_{13}^2 + 2 U_{23}^2 \right) .
\end{equation}
We can now see by inspection that the derivatives of the residual with respect to the off-diagonal elements are
\begin{equation}
\partialder{R}{U_{ij,i\neq j}} = 4 w U_{ij}.
\end{equation}
The derivatives of the residual with respect to the diagonal elements can be obtained as follows
\begin{align}
\partialder{R}{U_{11}} =& w \left( 2 (U_{11} - U_{iso})\partialder{(U_{11} - U_{iso})}{U_{11}}\right. \nonumber\\
                        &+ 2 (U_{22} - U_{iso})\partialder{(U_{22} - U_{iso})}{U_{22}}\nonumber\\
                        &+ 2 \left. (U_{33} - U_{iso})\partialder{(U_{33} - U_{iso})}{U_{33}} \right) \nonumber\\
                       =& w \left( 2 (U_{11} - U_{iso})(1 - \tfrac{1}{3}) + 2 (U_{22} - U_{iso})(-\tfrac{1}{3}) + 2 (U_{33} - U_{iso})(-\tfrac{1}{3}) \right) \nonumber\\
                       =& w \left( \tfrac{4}{3} U_{11} - \tfrac{2}{3} U_{22} - \tfrac{2}{3} U_{33}\right) \nonumber\\
                       =& 2 w (U_{11} - U_{iso}) .
\end{align}
This can be generalised as
\begin{equation}
\partialder{R}{U_{ii}} = 2 w (U_{ii} - U_{iso}) .
\end{equation}

\bibliography{cctbx_references}

\end{document}